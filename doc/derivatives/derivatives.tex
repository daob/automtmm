\documentclass[a4paper, 11pt]{article}

\usepackage[english]{babel}

%\usepackage{utopia}
\usepackage{times}

\usepackage{amsmath}
\usepackage{amssymb}

\title{Derivatives and standard errors of standardized parameters in structural
equation models}
\author{Daniel Oberski\\RECSM research paper, Universitat Pompeu Fabra}

\usepackage[margin=3.1cm]{geometry}

\bibliographystyle{apalike}
\usepackage{natbib}
\usepackage{siunitx} %for \num
\usepackage{rotating}

\newcommand{\n}{\eta}
\renewcommand{\l}{\lambda}
\renewcommand{\b}{\beta}
\newcommand{\p}{\phi}

\renewcommand{\d}{\,\mathrm{d}\,}

\newcommand{\definedas}{\triangleq}
\newcommand{\kronprod}{\otimes}
\newcommand{\hadaprod}{\circ}

\newcommand{\diag}{\mathrm{diag}}
\renewcommand{\vec}{\mathrm{vec}\,}
\newcommand{\vech}{\mathrm{vech}\,}

\newcommand{\Lambdastan}{\tilde{\Lambda}}
\newcommand{\Bstan}{\tilde{B}}
\newcommand{\thetastan}{\tilde{\theta}}

\newcommand{\0}{\boldsymbol{0}}

\newcommand{\var}{\mathrm{var}}

\newcommand{\R}{\texttt{R} 2.13.0 64-bit \citep{R}\;}
\newcommand{\lavaan}{\texttt{lavaan} 0.4.9 \citep{lavaan}\;}

\begin{document}
\maketitle

\begin{abstract}\noindent
In structural equation models, often not only the parameters of the model, but
also the standardized coefficients are of interest to the researcher. 
The literature does not, however, provide explicit analytical standard errors
for standardized parameters. %Due to this lack of availability of analytical derivatives in the literature, standard SEM software that wishes to provide SEM users with standard errors of the standardized parameters requires the use of numerical derivatives or approximate methods.

We derive an explicit expression in terms of the model parameters 
for derivatives and asymptotic standard errors of 
standardized parameters in structural equation models, that is straightforward to implement in standard SEM software. 
The expression is applied to an example application. 
\end{abstract}

\section{Introduction}\noindent

Linear structural equation models (SEM) with latent variables
have become a popular tool in different branches of science. Such models encompass as special cases a diverse range of common models of interest such as factor
analysis, multivariate regression, errors-in-variables models, growth, and
multilevel and multigroup models \citep{bollen1989structural}. In addition, extensions for categorical, count,
and censored dependent variables as well as complex sampling 
available \citep{muthen1995complex,muthen2002beyond}.

Although the main interest in such models lies in the parameters of 
the model, researchers' interest often also focuses on the so-called 
`standardized' parameters \citep{bollen1989structural}. Typical applications include examination of factor loadings and correlations in factor analysis, and the evaluation of the relative size of regression coefficients, possibly of latent variables.

In spite of the interest in standardized coefficients in structural equation models, standard errors and confidence intervals for these coefficients are usually not provided by the standard software\footnote{At the time of writing, the exception is Mplus version 5.2 and above \citep{muthen1998mplus}.}. To our knowledge the literature does not provide any explicit expression for the asymptotic standard errors of standardized coefficients. We remedy this situation by deriving an explicit expression for the asymptotic variance-covariance matrix of the standardized coefficients. The solution requires only the parameter estimates of the model and can be readily implemented in SEM software.

Section 2 provides a motivating example, using a SEM model with latent
variables where standardized parameters and their standard errors or confidence
intervals are of interest to the researcher.
Section 3 derives the explicit expression for the asymptotic variance-covariance matrix of the standardized estimates. In section 4, this expression is then applied to the example. Finally, the last section discusses the scope and limitations of this proposal and suggests future research.


\section{Example SEM with standardized parameters} 


The study of differences across societal groups in self-rated health is of interest to researchers in public health.   \cite{mackenbach}, 
for example, compare people with different incomes, age groups, and level of education.
It is well-known, however, that in order to be able to compare correlations across groups it is necessary for the reliability of 
the measures to be the same. Therefore not only across-group comparison of the levels of self-rated health are of interest, 
but also the evaluation of differences between the groups in reliability \citep{finnish study}. 

Different types of designs exist to estimate the reliability of survey measures. One such design is the repeated measures design, wherein the
same question is asked at least three times with a certain interval. One can then apply the so-called `quasi-simplex' model to estimate
reliability \cite{heise,wiley}. For an overview of other designs we refer to \cite{alwin}. 

The quasi-simplex model can be formulated and estimated as a structural equation model in different societal groups, such as groups with different levels of completed education. This model is shown for four repetitions in figure 
\ref{fig:model}. Following \cite{wiley}, the unstandardized error variances are restricted to be equal across the four repetitions. 
The model is formulated for the observed variables with loadings set to unity. The reliability coefficients of interest are then the standardized loadings.

We estimate this model using data from the LISS panel study in the Netherlands. The LISS panel is a random probability sample 
of 8000 Dutch citizens. The respondents answer questionnaires over the web. For more information about the LISS we refer to \cite{das}.
The LISS panel contains a study that included the commonly used self-rated health question,
\begin{quote}
	All things considered, how healthy would you say you are nowadays?
	1-5.
\end{quote}
This question was asked of 3425 LISS respondents in the years 2007, 2008, 2009, and 2010. Using the data from all four years, we estimate the 
quasi-simplex model shown in figure \ref{fig:model} as a SEM 
to yield four standardized loadings for each educational group, which can be interpreted as the reliability coefficients for each group\footnote{For the estimation we used
\R and \lavaan}. The results are shown in table \ref{tab:first-results}.

\begin{table}[ht]
\begin{center}
\begin{tabular}{lrrrrrrr}
  \hline  \hline
&  &\multicolumn{6}{c}{Education level}\\\cline{3-8}
%&	&	Primary		&VMBO		&		MBO	&HAVO/VWO&	HBO	&	WO	\\	
&	&	Primary		& Lower secondary & Middle secondary & Upper secondary & Lower tertiary & Upper tertiary\\
&	&	$n=279$		&$n=940$	&		$n=782$	&$n=369$	&	$n=799$	&	$n=256$	\\	
                                                                                  
	\hline              
\multicolumn{2}{l}{Year}\\                                                      
&	2007	&	0.799	&	0.821	&		0.822	&0.891	&	0.869	&	0.896	\\	
                                                                                  
&	2008	&	0.818	&	0.821	&		0.820	&0.879	&	0.878	&	0.898	\\	
	                                                                          
&	2009	&	0.828	&	0.812	&		0.808	&0.896	&	0.863	&	0.886	\\	
	                                                                          
&	2010	&	0.816	&	0.822	&		0.805	&0.877	&	0.874	&	0.887	\\	
  \hline     \hline
\end{tabular}
\caption{Reliability of self-rated health in the Netherlands 2007-2010 for groups with different levels of education.}\label{tab:first-results}
\end{center}
\end{table}

Table \ref{tab:first-results} shows that there appear to be quite some differences across the groups in reliability. 
For the lowest educational level, the reliability coefficient of the self-rated health question is 

\section{Standard errors of standardized parameters}

Let $y$ be a $p$-vector of observed variables, from which a sample is obtained.
The following SEM for $y$ is specified:
\begin{align}
\label{eq:lisrel_observed}
y &= \Lambda \n + \epsilon\\
\n &= B_0 \n + \zeta,\label{eq:lisrel_latent}
\end{align}
where $\n$ is a vector of unobserved variables, $\zeta$ is a vector of
disturbance terms and $\epsilon$ is a vector of measurement errors. Model
\ref{eq:lisrel_latent} implies the following model
$\Sigma_\n(\theta)$ for the variance-covariance matrix of the unobserved
variables as a function of a parameter vector $\theta$: 
\begin{equation}\label{eq:sigma_n}
    \Sigma_\n(\theta) = B^{-1} \Phi B^{-T},
\end{equation}
where $B \definedas I - B_0$ is positive definite, 
and $\Phi$ is the variance-covariance
matrix of $\zeta$. Model \ref{eq:lisrel_observed} can then be seen to imply the
following model $\Sigma_y(\theta)$ for the variance-covariance matrix of the
observed variables:
\begin{equation}\label{eq:sigma_y}
    \Sigma_y(\theta) = \Lambda B^{-1} \Phi B^{-T} \Lambda' + \Psi,
\end{equation}
where  $\Psi$ is the variance-covariance matrix of $\epsilon$. 
We assume throughout that both $\Sigma_y$ and $\Sigma_\n$ are positive
definite. In what follows we will write $\Sigma_.$ for $\Sigma_.(\theta)$ in
the
interest of clarity.

The parameters of the model are collected in a parameter
vector $$\theta \definedas [\vec{\Lambda}, \vech{\Phi}, \vec B_0, \vech
\Psi].$$
Often, interest focuses not only on the parameters $\theta$, 
but also on the so-called ``standardized'' matrices, which we denote
$\Lambdastan$ and $\Bstan_0$. These are defined as:
\begin{align}\label{eq:lambda_s}
    \Lambdastan &\definedas D^{-1}_y \Lambda D_\n
    \\
    \Bstan_0 &\definedas D^{-1}_\n B_0 D_\n,\label{eq:beta_s}
\end{align}
where $D_y \definedas \sqrt{I \hadaprod \Sigma_y}$, and 
$D_\n \definedas \sqrt{I \hadaprod \Sigma_\n}$. 

\marginpar{TODO: Define $\thetastan$.}

By standard application of the Delta method \citep[e.g.][]{oehlert1992note}, the asymptotic variance of $\thetastan$ is 
\begin{equation}\label{eq:deltamethod}
	\left(\frac{d \thetastan}{d \theta}\right) 
		\var(\theta) 
	\left(\frac{d \thetastan}{d \theta}\right)',
\end{equation}
where $\var(\theta)$ is the appropriate asymptotic variance matrix of the free model parameters $\theta$ \cite[e.g.][]{satorra1989alternative}.


\subsection{Derivatives of the standardized parameters}

To obtain the full expression for the variance-covariance matrix of
standardized parameters in equation \ref{eq:deltamethod}, the needed
derivatives of the standardized parameters with respect to the free parameters
of the model are derived here.

From definition \ref{eq:lambda_s}, 
\begin{multline}\label{eq:dveclam}
\d\vec\Lambdastan = 
    (D_\n \Lambda' \kronprod I_p) \d \vec D_y^{-1} + 
    (D_\n \kronprod D_y^{-1}) \d\vec\Lambda + \\
    (I_q \kronprod D_y^{-1} \Lambda) \d\vec D_\n,
\end{multline}
and from definition \ref{eq:beta_s}, 
\begin{multline}\label{eq:dvecbeta}
\d\vec \Bstan_0 = 
    (D_\n B_0' \kronprod I_p) \d \vec D_\n^{-1} + 
    (D_\n \kronprod D_\n^{-1}) \d\vec B_0 + \\
    (I_q \kronprod D_\n^{-1} B_0) \d\vec D_\n.
\end{multline}

The differentials of the standardized parameter matrices are, thus, 
functions of the differentials of the covariance structure models $\Sigma_y$
and
$\Sigma_\n$.
\cite{neudecker1991linear} derived the differential of the implied variance
matrix $\Sigma_y$ of the observed variables. To ensure completeness of the treatment, we repeat it here:
\begin{equation}\label{eq:delta_y}
\begin{split}
\d\vec \Sigma_y = (I + K) (\Lambda B^{-1} \Phi B^{-T} \kronprod I) 
\d\vec\Lambda 
+ \\
(I + K) (\Lambda B^{-1} \Phi B^{-T} \kronprod \Lambda B^{-1}) \d\vec B_0
+ \\
(\Lambda B^{-1} \kronprod \Lambda B^{-1}) P \d \vech \Phi
+ \\
\d \vech \Psi,
\end{split}
\end{equation}
where the commutation matrix $K$, the duplication matrix $P$, and the operators
$\vec$ and $\vech$ are defined in \cite{magnus1988matrix}
Let $\Delta_\l$ equal the coefficient matrix
 of $\d\vec\Lambda$ in equation \ref{eq:delta_y}, 
$\Delta_\b$ the coefficient  of $\d\vec B_0$, and
$\Delta_\p$ the coefficient  of $\d\vech\Phi$.


The differential of the variance matrix of $\n$ can be obtained as:
\begin{equation}\label{eq:delta_n}
\begin{split}
\d\vec \Sigma_\n = 
(I + K) (B^{-1} \Phi B^{-T} \kronprod  B^{-1}) \d\vec B_0
+ \\
(B^{-1} \kronprod  B^{-1}) P \d \vech \Phi.
\end{split}
\end{equation}
Let $\Delta^*_\b$ and $\Delta^*_\p$  be the coefficients of $\d\vec B_0$ and
$\d\vech\Phi$ respectively in equation \ref{eq:delta_n}.

Also, let
\begin{align}
    C_{y2} \definedas&
        - (I_p \kronprod \frac{1}{2 (I_p \hadaprod \Sigma_y)^{\frac{3}{2}}})
        \diag[\vec(I_p)],
\\
    C_{\n1} \definedas&
        (I_q \kronprod \frac{1}{2 (I_q \hadaprod \Sigma_\n)^{\frac{1}{2}}})
        \diag[\vec(I_q)],
\\
    C_{\n2} \definedas&
        - (I_q \kronprod \frac{1}{2 (I_q \hadaprod \Sigma_\n)^{\frac{3}{2}}})
        \diag[\vec(I_q)].
\end{align}

Then, applying standard operations on \ref{eq:dveclam} and rearranging terms,
the differential of the standardized $\Lambdastan$ matrix is
\begin{equation}\label{eq:dveclam_final}
\begin{split}
  \d\vec\Lambdastan = 
     & [(D_\n \Lambda' \kronprod I_p) C_{y2} \Delta_\l + 
        (D_\n \kronprod D_y^{-1})] 
        \d\vec\Lambda +\\
     & [(D_\n \Lambda' \kronprod I_p) C_{y2} \Delta_\b + 
        (I_q \kronprod D_y^{-1}\Lambda) C_{\n1} \Delta^*_\b ] 
        \d\vec B_0 +\\
     & [(D_\n \Lambda' \kronprod I_p) C_{y2} \Delta_\p + 
        (I_q \kronprod D_y^{-1}\Lambda) C_{\n1} \Delta^*_\p ] 
        \d\vec \Phi +\\
     & (D_\n \Lambda' \kronprod I_p) C_{y2} 
        \d\vec \Psi.
\end{split}\end{equation}
And similarly, the differential of the standardized $\Bstan_0$ matrix from
equation \ref{eq:dvecbeta} is
\begin{equation}\label{eq:dvecbeta_final}
\begin{split}
  \d\vec \Bstan_0 =
     & [\{(D_\n B_0' \kronprod I_q) C_{\n2}  + 
        (I_q \kronprod D_\n^{-1}B_0) C_{\n1} \} \Delta^*_\b +
           (D_\n \kronprod D_\n^{-1})  ] 
        \d\vec B_0 +\\
     & [(D_\n B_0' \kronprod I_q) C_{\n2}  + 
        (I_q \kronprod D_\n^{-1}B_0) C_{\n1} ] \Delta^*_\p 
        \d\vec \Phi.
\end{split}\end{equation}


\marginpar{TODO: Lose the $G$? Just write $d\thetastan/d\theta$?}

From the differentials in equations \ref{eq:dveclam_final} and
\ref{eq:dvecbeta_final},
we conclude that the derivative matrices  $G_{\tilde{\l}}$ and 
$G_{\tilde{\b}}$ of the standardized parameters $\vec \Bstan$ and 
$\vec \Lambdastan$ with respect to the free parameters of the model 
$\theta$ will be the partitioned matrices
\begin{multline}
    G_{\tilde{\l}} = [
        (D_\n \Lambda' \kronprod I_p) C_{y2} \Delta_\l + 
            (D_\n \kronprod D_y^{-1}), \\
        (D_\n \Lambda' \kronprod I_p) C_{y2} \Delta_\p +
         (I_q \kronprod D_y^{-1}\Lambda) C_{\n1} \Delta^*_\p, \\
        (D_\n \Lambda' \kronprod I_p) C_{y2} \Delta_\b + 
            (I_q \kronprod D_y^{-1}\Lambda) C_{\n1} \Delta^*_\b,\\
        (D_\n \Lambda' \kronprod I_p) C_{y2}  
 ] 
\end{multline}
and
\begin{multline}
    G_{\tilde{\b}} = [
        \0,
        \{(D_\n B_0' \kronprod I_q) C_{\n2}  + 
            (I_q \kronprod D_\n^{-1}B_0) C_{\n1} \} \Delta^*_\p
            ,\\
        \{(D_\n B_0' \kronprod I_q) C_{\n2}  + 
            (I_q \kronprod D_\n^{-1}B_0) C_{\n1} \} \Delta^*_\b +
            (D_\n \kronprod D_\n^{-1})
            ,
        \0
 ] .
\end{multline}

\marginpar{TODO: Add $\Phi$ and $\Psi$; put it all together in one matrix}

\section{Application: standard errors of standardized parameters}

\begin{table}[ht]
\begin{center}
\begin{tabular}{lrrrrrrr}
  \hline  \hline
&  &\multicolumn{6}{c}{Education level}\\\cline{3-8}
&	&	Primary		&VMBO		&		MBO	&HAVO/VWO&	HBO	&	WO	\\	
&	&	$n=279$		&$n=940$	&		$n=782$	&$n=369$	&	$n=799$	&	$n=256$	\\	
                                                                                  
	\hline              
\multicolumn{2}{l}{Year}\\                                                      
&	2007	&	0.799	&	0.821	&		0.822	&0.891	&	0.869	&	0.896	\\	
&		&	(0.029)	&	(0.014)	&		(0.015)	&(0.016)	&	(0.013)	&	(0.017)	\vspace{7pt}\\	
                                                                                  
&	2008	&	0.818	&	0.821	&		0.820	&0.879	&	0.878	&	0.898	\\	
&		&	(0.024)	&	(0.013)	&		(0.014)	&(0.017)	&	(0.012)	&	(0.017)	\vspace{7pt}\\	
	                                                                          
&	2009	&	0.828	&	0.812	&		0.808	&0.896	&	0.863	&	0.886	\\	
&		&	(0.023)	&	(0.014)	&		(0.015)	&(0.015)	&	(0.013)	&	(0.018)	\vspace{7pt}\\	
	                                                                          
&	2010	&	0.816	&	0.822	&		0.805	&0.877	&	0.874	&	0.887	\\	
&		&	(0.027)	&	(0.014)	&		(0.017)	&(0.018)	&	(0.013)	&	(0.019)	\\	
  \hline     \hline
\end{tabular}
\caption{Reliability of self-rated health in the Netherlands 2007-2010 for different educational groups. 
Asymptotic standard errors based on the method described in this paper are also given. For direct tests of 
differences between the groups in reliability, see table \ref{tab:tests}.}\label{tab:health-education}
\end{center}
\end{table}


% latex table generated in R 2.13.0 by xtable 1.5-6 package
% Sun May 15 22:42:47 2011
\begin{table}[ht]
\begin{center}
\begin{tabular}{lrrrl}
  \hline  \hline
 Comparison & $t_{\mathrm{dif}}$ & $p_{\mathrm{dif}}$ & $p_{\mathrm{dif,adj}}$ &  \\ 
  \hline
Primary $<$$>$ vmbo & -0.68 & 0.25 & 0.96 &  \\ 
Primary $<$$>$ havo/vwo & -2.76 & 0.00 & 0.03 & * \\ 
Primary $<$$>$ mbo & -0.71 & 0.24 & 0.96 &  \\ 
Primary $<$$>$ hbo & -2.18 & 0.01 & 0.10 &  \\ 
Primary $<$$>$ wo & -2.86 & 0.00 & 0.02 & * \\ 
vmbo $<$$>$ havo/vwo & -3.25 & 0.00 & 0.01 & * \\ 
vmbo $<$$>$ mbo & -0.06 & 0.48 & 0.96 &  \\ 
vmbo $<$$>$ hbo & -2.46 & 0.01 & 0.06 &  \\ 
vmbo $<$$>$ wo & -3.35 & 0.00 & 0.01 & * \\ 
 havo/vwo $<$$>$ mbo & 3.12 & 0.00 & 0.01 & * \\ 
 havo/vwo $<$$>$ hbo & 1.07 & 0.14 & 0.71 &  \\ 
 havo/vwo $<$$>$ wo & -0.22 & 0.41 & 0.96 &  \\ 
 mbo $<$$>$ hbo & -2.33 & 0.01 & 0.08 &  \\ 
 mbo $<$$>$ wo & -3.22 & 0.00 & 0.01 & * \\ 
 hbo $<$$>$ wo & -1.26 & 0.10 & 0.62 &  \\ 
   \hline  \hline
\end{tabular}\caption{Multiple comparison between educational groups for the reliability in 2007. 
	The groups that are compared on reliability are shown in the first column. The second column shows the
	$t$-value for the test of no difference between the groups in reliability. The last two columns show 
	the corresponding $p$-value and $p$-value adjusted for multiple comparisons \citep{holm1979simple}, respectively.}\label{tab:tests}
\end{center}
\end{table}


\section{Discussion}

\subsection{Scope}

Limited, categorical variables, effects/correlations between latent variables,
factor loadings, error correlations.

\subsection{Others}
\cite{bollen1990direct} note that an approximate method exists for regression
analysis that does not take the estimation of the model-implied standard
deviations into account. 

References to Delta method but without providing the derivatives.

Likelihood based confidence intervals, bootstrapping.

Our method dispenses with the need for approximate methods or numerical
derivatives.

\subsection{Future}

Examine relative performance of confidence intervals, similar to work done for
indirect effects.


\bibliography{derivatives}

\appendix

\section{Calculations for the example analysis}


\begin{table}
\begin{tabular}{lrrrrrrrrrrrr}\hline\hline
	& \multicolumn{11}{c}{Unstandardized free parameters $\theta$}\\
	%\cline{2-12}
			& $\psi_{11}$ & $\psi_{22}$ &$\psi_{33}$ &$\psi_{44}$
			& $\phi_{11}$ & $\phi_{22}$ &$\phi_{33}$ &$\phi_{44}$ 
			& $\beta_{21}$ & $\beta_{32}$ &$\beta_{43}$
			\\
\hline
%\multicolumn{4}{l}{$d \vec \Lambda_s / d \vec \theta$} &\\
                   \\                                                      
$\lambda_{s, 11}$ &-0.733  & .      & .      & .     & 0.277 & .     & .     & .     & .     & .     & .   & \\
$\lambda_{s, 22}$&.     & -0.739  & .      & .     & 0.217 & 0.282 & .     & .     & 0.209 & .     & .   & \\
$\lambda_{s, 33}$&.      & .     & -0.755  & .     & 0.188 & 0.244 & 0.300 & .     & 0.180 & 0.225 & .   & \\
$\lambda_{s, 33}$&.      & .      & .     & -0.754 & 0.173 & 0.225 & 0.276 & 0.298 & 0.166 & 0.207 & 0.230\\
\\
%\multicolumn{4}{l}{$d \vec B_s / d \vec \theta$} &\\
             \\                                            
$\beta_{s, 21}$ &. & . & . & . & 0.232 & -1.059  & .      & .     & 0.223 & .     & .    & \\
$\beta_{s, 32}$&. & . & . & . & 0.131  & 0.171 & -1.147  & .     & 0.126 & 0.157 & .    & \\
$\beta_{s, 43}$&. & . & . & . & 0.058  & 0.076  & 0.093 & -1.193 & 0.056 & 0.070 & 0.077\\
\hline\hline
\end{tabular}
\caption{Derivatives of the standardized parameters $\Lambda_s$ and $B_s$ with respect
to the free parameters of the model in the quasi simplex example.}
\label{tab:jacobian}
\end{table}


\begin{table}
\begin{small}
\begin{tabular}{rrrrrrrrr}\hline\hline
	&$\psi$ & $\phi_{11}$ & $\phi_{22}$ &$\phi_{33}$ &$\phi_{44}$ & $\beta_{21}$ & $\beta_{32}$ &$\beta_{43}$  \\
\hline
  $\psi$   & \num{0.2e-4}\\
  $\phi_{11}$  & -\num{0.2e-4}  & \num{2.2e-4}\\
  $\phi_{22}$  & -\num{0.3e-4}  & \num{0.4e-4}  & \num{0.9e-4}\\
  $\phi_{33}$  & -\num{0.2e-4}  & \num{0.2e-4}  & \num{0.1e-4}  & \num{0.5e-4}\\
  $\phi_{44}$  & -\num{0.3e-4}  & \num{0.3e-4}  & \num{0.3e-4}  & \num{0.1e-4}  & \num{0.8e-4}\\
  $\beta_{21}$  & \num{0.4e-4} & -\num{1.5e-4} & -\num{0.9e-4} & -\num{0.1e-4} & -\num{0.5e-4}  & \num{4.3e-4}\\
  $\beta_{32}$  & \num{0.2e-4} & -\num{0.2e-4} & -\num{0.4e-4} & -\num{0.4e-4} & -\num{0.1e-4} & -\num{0.9e-4}   & \num{3.3e-4}\\
  $\beta_{43}$  & \num{0.3e-4} & -\num{0.3e-4} & -\num{0.3e-4} & -\num{0.3e-4} & -\num{0.7e-4}  & \num{0.4e-4}  & -\num{1.0e-4}   & \num{3.4e-4}\\
\hline\hline
\end{tabular}\end{small}
\caption{Variance-covariance matrix of the parameter estimates.}
\label{tab:vcov}
\end{table}

\vspace{12pt}

\begin{table}\begin{center}
\begin{tabular}{rrrr}\hline\hline
 Parameter & 	Estimate &	s.e. &	$z$\\
 \hline
$\lambda_{s,11}$ & 0.852& 0.006 &  134\\
$\lambda_{s,22}$ &0.851 &0.006  & 139\\
$\lambda_{s,33}$ &0.846 &0.006  & 136\\
$\lambda_{s,44}$ &0.847& 0.007  & 128\\
$\beta_{s,21}$& 0.882           &0.012 & 75\\
$\beta_{s,32}$& 0.920            &0.009 &103\\
$\beta_{s,43}$& 0.960            &0.011  &85\\
\hline\hline
\end{tabular}
\caption{Standardized parameter estimates and asymptotic standard errors for the example ($n = 3425$).}
\label{tab:vcov}\end{center}
\end{table}



\section{Extension to multigroup SEM}

In multigroup SEM, let each matrix in the above equations equal the block
diagonal of the corresponding matrices in each group. 

Then define the stacking matrix $Z_{p \times pg} \definedas \left[I_p, I_p,
..., I_p\right]$. The derivatives are then 
$$
	(I_q \kronprod Z) G
$$

\end{document}
