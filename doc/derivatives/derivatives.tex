\documentclass{article}

\usepackage{amsmath}
\usepackage{amssymb}

\title{Derivatives and standard errors of standardized parameters in the LISREL
model}
\author{Daniel Oberski\\RECSM research paper, Universitat Pompeu Fabra}


\newcommand{\n}{\eta}
\renewcommand{\l}{\lambda}
\renewcommand{\b}{\beta}
\newcommand{\p}{\phi}

\renewcommand{\d}{\,\mathrm{d}\,}

\newcommand{\definedas}{\triangleq}
\newcommand{\kronprod}{\otimes}
\newcommand{\hadaprod}{\circ}

\newcommand{\diag}{\mathrm{diag}}
\renewcommand{\vec}{\mathrm{vec}\,}
\newcommand{\vech}{\mathrm{vech}\,}


\begin{document}
\maketitle

Assume the following model for a vector of observed variables $y$ has been
specified:
\begin{align}
\label{eq:lisrel_observed}
y &= \Lambda \n + \epsilon\\
\n &= B_0 \n + \zeta,\label{eq:lisrel_latent}
\end{align}
where $\n$ is a vector of unobserved variables, $\zeta$ is a vector of
disturbance terms and $\epsilon$ is a vector of measurement errors. Model
\ref{eq:lisrel_latent} implies the following model
$\Sigma_\n$ for the variance-covariance matrix of the unobserved variables:
\begin{equation}\label{eq:sigma_n}
    \Sigma_\n = B^{-1} \Phi B^{-T},
\end{equation}
where $B \definedas I - B_0$ is positive definite, 
and $\Phi$ is the variance-covariance
matrix of $\zeta$. Model \ref{eq:lisrel_observed} can then be seen to imply the
following model $\Sigma_y$ for the variance-covariance matrix of the observed
variables:
\begin{equation}\label{eq:sigma_y}
    \Sigma_y = \Lambda B^{-1} \Phi B^{-T} \Lambda' + \Psi,
\end{equation}
where  $\Psi$ is the variance-covariance matrix of $\epsilon$. 
We assume throughout that both $\Sigma_y$ and $\Sigma_\n$ are positive
definite.

Often, interest focuses not only on the parameter matrices $\Lambda$, $B_0$, 
$\Phi$, and $\Psi$, but also on the so-called ``standardized'' matrices
$\Lambda^{(s)}$ and $B^{(s)}_0$. These are defined as:
\begin{align}\label{eq:lambda_s}
    \Lambda^{(s)} &\definedas D^{-1}_y \Lambda D_\n
    \\
    B_0^{(s)} &\definedas D^{-1}_\n B_0 D_\n,\label{eq:beta_s}
\end{align}
where $D_y \definedas \sqrt{I \hadaprod \Sigma_y}$, and 
$D_\n \definedas \sqrt{I \hadaprod \Sigma_\n}$. We now derive the differentials
of these standardized parameter matrices. 

From definition \ref{eq:lambda_s}, 
\begin{multline}
\d\vec\Lambda^{(s)} = 
    (D_\n \Lambda' \kronprod I_p) \d \vec D_y^{-1} + 
    (D_\n \kronprod D_y^{-1}) \d\vec\Lambda + \\
    (I_q \kronprod D_y^{-1} \Lambda) \d\vec D_\n.
\end{multline}

The differentials of the standardized parameter matrices are, thus, 
functions of the differentials of the covariance structure models $\Sigma_y$ and
$\Sigma_\n$.
From Neudecker \& Satorra (1990), the differential of the implied variance
matrix $\Sigma_y$ of the observed variables is:
\begin{equation}\label{eq:delta_y}
\begin{split}
\d\vec \Sigma_y = (I + K) (\Lambda B^{-1} \Phi B^{-T} \kronprod I) 
\d\vec\Lambda 
- \\
(I + K) (\Lambda B^{-1} \Phi B^{-T} \kronprod \Lambda B^{-1}) \d\vec B
+ \\
(\Lambda B^{-1} \kronprod \Lambda B^{-1}) E \d \vech \Phi
+ \\
\d \vech \Psi,
\end{split}
\end{equation}
where the commutation matrix $K$, the elimination matrix $E$, and the operators
$\vec$ and $\vech$ are defined in Magnus and Neudecker (1989).
Let $\Delta_\l$ equal the coefficient matrix
 of $\d\vec\Lambda$ in equation \ref{eq:delta_y}, 
$\Delta_\b$ the coefficient  of $\d\vec B$, and
$\Delta_\p$ the coefficient  of $\d\vech\Phi$.


From equation \ref{eq:delta_y} together with equation \ref{eq:sigma_n} the
differential of the variance matrix of $\n$ can be obtained as:
\begin{equation}\label{eq:delta_n}
\begin{split}
\d\vec \Sigma_\n = 
- (I + K) (B^{-1} \Phi B^{-T} \kronprod  B^{-1}) \d\vec B
+ \\
(B^{-1} \kronprod  B^{-1}) E \d \vech \Phi.
\end{split}
\end{equation}
Let $\Delta^*_\b$ and $\Delta^*_\p$  be the coefficients of $\d\vec B$ and
$\d\vech\Phi$ respectively in equation \ref{eq:delta_n}.


Let 
\begin{align}
    C_{y2} \definedas&
        (I_p \kronprod \frac{1}{2 (I_p \hadaprod \Sigma_y)^{\frac{3}{2}}})
        \diag[\vec(I_p)],
\\
    C_{\n1} \definedas&
        (I_q \kronprod \frac{1}{2 (I_q \hadaprod \Sigma_\n)^{\frac{1}{2}}})
        \diag[\vec(I_q)],
\\
    C_{\n2} \definedas&
        (I_q \kronprod \frac{1}{2 (I_q \hadaprod \Sigma_\n)^{\frac{3}{2}}})
        \diag[\vec(I_q)].
\end{align}


\end{document}
