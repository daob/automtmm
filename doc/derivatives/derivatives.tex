\documentclass[a4paper, 11pt]{article}

\usepackage[english]{babel}

%\usepackage{utopia}

\usepackage{color}
\definecolor{darkblue}{rgb}{0, 0, 0.4}

\usepackage{times}
\usepackage[colorlinks=true,linkcolor=darkblue,urlcolor=darkblue,citecolor=darkblue]{hyperref}

\usepackage{amsmath}
\usepackage{amssymb}

\title{Derivatives and standard errors of standardized parameters in structural
equation models}
\author{Daniel Oberski\\RECSM research paper, Universitat Pompeu Fabra}

\usepackage[margin=3.1cm]{geometry}

\usepackage{tabularx}

\bibliographystyle{apalike}
\usepackage{natbib}
\usepackage{siunitx} %for \num
\usepackage{rotating}

\newcommand{\n}{\eta}
\renewcommand{\l}{\lambda}
\renewcommand{\b}{\beta}
\newcommand{\p}{\phi}

\renewcommand{\d}{\,\mathrm{d}\,}

\newcommand{\definedas}{\triangleq}
\newcommand{\kronprod}{\otimes}
\newcommand{\hadaprod}{\circ}

\newcommand{\diag}{\mathrm{diag}}
\renewcommand{\vec}{\mathrm{vec}\,}
\newcommand{\vech}{\mathrm{vech}\,}

\newcommand{\Lambdastan}{\tilde{\Lambda}}
\newcommand{\Bstan}{\tilde{B}}
\newcommand{\thetastan}{\tilde{\theta}}

\newcommand{\0}{\boldsymbol{0}}

\newcommand{\var}{\mathrm{var}}

\newcommand{\R}{\texttt{R} 2.13.0 64-bit \citep{R}\;}
\newcommand{\lavaan}{\texttt{lavaan} 0.4.9 \citep{lavaan}\;}

\begin{document}
\maketitle

\begin{abstract}\noindent
In structural equation models (SEM), often not only the standard errors of parameters of the model, but
also of the standardized coefficients are of interest to the researcher. 
Examples are the comparison of reliability coefficients across different groups, and meta-analysis of 
standardized quantities.
The literature does not, however, provide explicit analytical standard errors 
for standardized parameters. %Due to this lack of availability of analytical derivatives in the literature, standard SEM software that wishes to provide SEM users with standard errors of the standardized parameters requires the use of numerical derivatives or approximate methods.

We derive an explicit expression in terms of the model parameters 
for derivatives and the asymptotic variance-covariance matrix of 
standardized parameters in structural equation models, that is straightforward to implement in standard SEM software. 
The solution is applied to an analysis of the reliability of self-rated health in groups with different levels of education.
\end{abstract}

\section{Introduction}\noindent
Linear structural equation models (SEM) with latent variables
have become a popular tool in the behavioral sciences. Such models encompass as special cases a diverse range of common models of interest such as factor
analysis, multivariate regression, errors-in-variables models, growth, and
multilevel and multigroup models \citep{bollen1989structural}. Extensions are available for categorical, count,
and censored dependent variables as well as complex sampling 
 \citep{muthen1995complex,muthen2002beyond}.

Although the main interest in such models lies in the parameters of 
the model, researchers' interest often also focuses on the so-called 
`standardized' parameters \citep{bollen1989structural}. Typical applications include examination of factor loadings and correlations in factor analysis, as well as the evaluation of the relative size of regression coefficients, possibly of latent variables.

In spite of the interest in standardized coefficients in structural equation models, to our knowledge the literature does not provide any explicit expression for the asymptotic standard errors of standardized coefficients. As a consequence, standard errors and confidence intervals for these coefficients are usually not provided by the standard software\footnote{At the time of writing, the exception is Mplus version 5.2 and above \citep{muthen1998mplus}.}.  We remedy this situation by deriving an explicit expression, in terms of the unstandardized parameter estimates, for the asymptotic variance-covariance matrix of the standardized coefficients. The solution requires only the parameter estimates of the model and can be readily implemented in SEM software.

Section 2 provides a motivating example, using a SEM model with latent
variables where standardized parameters and their standard errors and confidence
intervals are of interest to the researcher.
Section 3 derives the explicit expression for the asymptotic variance-covariance matrix of the standardized estimates. Section 4 applies this expression to the example. The last section discusses the scope and limitations of this proposal and suggests future research.


\section{Example SEM with interest in standardized parameters} 


The study of differences across societal groups in self-rated health is of interest to researchers in public health.   For example,  \cite{von2006education} 
compare people with different incomes, age groups, and level of education.
It is well-known, however, that in order to be able to compare correlations across groups it is necessary for the reliability of 
the measures to be the same \citep[e.g.][]{saris_design_2007}. Therefore not only across-group comparison of the levels of self-rated health are of interest, 
but also the evaluation of differences between the groups in reliability \citep{lundberg1996assessing}. 

Different types of designs exist to estimate the reliability of survey measures. One such design is the repeated measures design, wherein the
same question is asked at least three times with a certain interval. One can then apply the so-called `quasi-simplex' model in different groups to the estimate
reliability, and compare reliabilities across groups \citep{heise1970validity,wiley35wiley}. For an overview of other designs we refer to \cite{alwin_margins_2007}. 

The quasi-simplex model can be formulated and estimated as a multiple-group structural equation model for groups with different  levels of highest completed education. This model is shown for four repetitions in figure 
\ref{fig:model}. Following \cite{wiley35wiley}, the unstandardized error variances are restricted to be equal across the four repetitions. The reliability coefficients of interest are then the standardized loadings. This yields the  structural equation model shown as a path diagram in figure \ref{fig:model}.
\if 1=2
\begin{eqnarray*}\begin{split}
\textbf{y} =  \textbf{I} {\eta} + \epsilon,& &
{\eta} = \begin{pmatrix}
	0 & 0 & 0 & 0\\
	\beta_{21} & 0 & 0 & 0\\
	0 & \beta_{32} & 0 & 0\\
	0 & 0 & \beta_{43} & 0\\	
\end{pmatrix} {\eta} + \zeta
\end{split}
\end{eqnarray*}
\fi


\begin{figure}[bt]\begin{center}
\caption{Quasi-simplex model for fours repeated measures of self-rated health in the LISS panel 2007--2010.}
\label{fig:model}
\includegraphics[width=.9\textwidth]{self-rated-health-quasi-simplex}
\end{center}
\end{figure}

We estimate this model using data from the LISS panel study in the Netherlands. The LISS panel is a random probability sample 
of 8000 Dutch citizens. The respondents answer questionnaires over the web. For more information about the LISS we refer to \cite{scherpenzeel2011data}.
The  panel contains a study that included the commonly used self-rated health question\footnote{The original question was asked in Dutch. See \url{http://www.lissdata.nl/dataarchive/question_constructs/view/600}}. The question was asked as follows:
\begin{quote}
	How would you describe your health, generally speaking?
	
	\begin{enumerate}  \setlength{\itemsep}{0pt}  \setlength{\parskip}{0pt}
  \setlength{\parsep}{0pt}
		\item Poor
		\item Moderate
		\item Good
		\item Very good
		\item Excellent
	\end{enumerate}
\end{quote}


\begin{table}[bt]\begin{small}
\begin{center}
\begin{tabular}{llrrrrrrrrr}  \hline  \hline
&&&  \multicolumn{8}{c}{\emph{Year}}\\\cline{4-11}
&&& \multicolumn{2}{c}{2007} & \multicolumn{2}{c}{2009} & \multicolumn{2}{c}{2009} & \multicolumn{2}{c}{2010} \\
&&$n$&Mean&sd&Mean&sd&Mean&sd&Mean&sd\\
%&	&	Primary		&VMBO		&		MBO	&HAVO/VWO&	HBO	&	WO	\\	
  \hline
  \multicolumn{2}{l}{Education level}\\
 
& Primary	  & $279$ & 2.83 & (0.69) & 2.91 & (0.73) & 2.84 & (0.74) & 2.82 & (0.72) \\ 
& Lower secondary  & $940$ & 2.98 & (0.73) & 3.08 & (0.73) & 3.01 & (0.71) & 2.97 & (0.73) \\ 
& Middle secondary & $782$ & 3.21 & (0.80) & 3.24 & (0.76) & 3.27 & (0.81) & 3.20 & (0.75) \\ 
& Upper secondary  & $369$ & 3.17 & (0.75) & 3.18 & (0.74) & 3.16 & (0.72) & 3.10 & (0.72) \\ 
& Lower tertiary	  & $799$ & 3.28 & (0.73) & 3.28 & (0.76) & 3.22 & (0.72) & 3.22 & (0.74) \\ 
& Upper tertiary	  & $256$ & 3.33 & (0.86) & 3.35 & (0.86) & 3.32 & (0.82) & 3.29 & (0.83) \\ 
   \hline
   \hline
\end{tabular}
\caption{Means and standard deviations (in brackets) of the self-rated health question across the four repetitions, in 
groups with different levels of education.}\label{tab:descriptives}
\end{center}\end{small}
\end{table}

This question was asked of 3425 LISS respondents in the years 2007, 2008, 2009, and 2010. Splitting the sample into
groups with different levels of education, the means and standard deviations shown in table \ref{tab:descriptives} are obtained.
Using the data from all four years, we estimate the  quasi-simplex model shown in figure \ref{fig:model} as a SEM
to yield four standardized loadings for each educational group, which can be interpreted as the reliability coefficients for each group\footnote{For the estimation we used \R and \lavaan}. The unstandardized parameter estimates and model fit measures are shown in table \ref{tab:results-unstandardized}, while standardized loadings (reliability coefficients) are shown in table \ref{tab:first-results}.

\begin{table}\begin{small}
\begin{tabularx}{\textwidth}{l*{8}{X}}
  \hline  \hline
  	& \multicolumn{6}{c}{Group: education level}\\	\cline{2-7}
    Par.  &       Primary         & Lower secondary & Middle secondary & Upper secondary & Lower tertiary & Upper tertiary\\
  \hline
$\phi_{11}$   & 0.50 (0.049) & 0.59 (0.068) & 0.35 (0.026) & 0.38 (0.030) & 0.31 (0.044) & 0.40 (0.029) \\ 
$\phi_{22}$   & 0.13 (0.027) & 0.17 (0.040) & 0.10 (0.018) & 0.03 (0.019) & 0.05 (0.031) & 0.12 (0.021) \\ 
$\phi_{33}$   & 0.10 (0.026) & 0.08 (0.027) & 0.05 (0.012) & 0.04 (0.013) & 0.02 (0.024) & 0.08 (0.015) \\ 
$\phi_{44}$   & 0.06 (0.027) & 0.06 (0.035) & -0.02 (0.017) & 0.02 (0.019) & 0.02 (0.030) & 0.08 (0.020) \\ 
$\beta_{21}$  & 0.78 (0.051) & 0.85 (0.062) & 0.84 (0.045) & 0.96 (0.050) & 0.99 (0.098) & 0.89 (0.043) \\ 
$\beta_{32}$  & 0.99 (0.055) & 0.87 (0.052) & 0.89 (0.039) & 0.89 (0.041) & 1.01 (0.074) & 0.83 (0.034) \\ 
$\beta_{43}$  & 0.84 (0.044) & 0.95 (0.057) & 1.07 (0.041) & 0.96 (0.044) & 0.92 (0.065) & 0.94 (0.039) \\ 
$\psi_{ii}$   & 0.13 (0.015) & 0.14 (0.019) & 0.17 (0.010) & 0.18 (0.010) & 0.17 (0.017) & 0.13 (0.010) \\ 
     \hline  \hline
\end{tabularx}\caption{Unstandardized parameter estimates and standard errors for the 
multiple group quasi-simplex model shown in figure \ref{fig:model}.
Chi-square:  6.923 on  12 degrees of freedom  ($p = 0.863$),  SRMR:  0.0041, RMSEA: 0.000}
\label{tab:results-unstandardized}
\end{small}\end{table}

\begin{table}[bth]
\begin{center}\begin{small}
\begin{tabular}{lllrrrr}  \hline  \hline
&&&  \multicolumn{4}{c}{\emph{Year}}\\\cline{4-7}
&&$n$& 2007&2008&2009&2010\\
%&	&	Primary		&VMBO		&		MBO	&HAVO/VWO&	HBO	&	WO	\\	
  \hline
  \multicolumn{2}{l}{Education level}\\
& Primary	   & $279$  & 0.799 & 0.818 & 0.828 & 0.816 \\ 
& Lower secondary  & $940$  & 0.821 & 0.821 & 0.812 & 0.822 \\ 
& Middle secondary & $782$  & 0.891 & 0.879 & 0.896 & 0.877 \\ 
& Upper secondary  & $369$  & 0.822 & 0.820 & 0.808 & 0.805 \\ 
& Lower tertiary   & $799$  & 0.869 & 0.878 & 0.863 & 0.874 \\ 
& Upper tertiary   & $256$  & 0.896 & 0.898 & 0.886 & 0.887 \\ 
  \hline     \hline
\end{tabular}
\caption{Reliability (standardized $\Lambda$) of self-rated health in the Netherlands 2007-2010 for groups with different levels of education.}\label{tab:first-results}\end{small}
\end{center}
\end{table}

Table \ref{tab:first-results} shows that there appear to be quite some differences across the groups in reliability. 
For the lowest educational level in 2007, the reliability coefficient of self-rated health is 0.799, while for the highest level it is 0.896.
There also appear to be some differences between the years, although the differences across years are much smaller than those across
different educational groups. Finally, the values of all reliability coefficients in table \ref{tab:first-results} are quite a bit higher than those reported by \cite{lundberg1996assessing} 
for self-rated health. This may be due to differences in the question and/or mode of interviewing, but it may also be due to sampling. 

One would require standard errors or confidence intervals to properly study the differences between people with different levels of attained education in reliability. The same is true if one would like to analyze these results together with those of \cite{lundberg1996assessing}  as a meta-analysis.
It is clear, therefore, that to answer the questions that are of interest to the researcher, standard errors and confidence intervals for the 
reliability coefficients would be useful. The next section derives these standard errors for general structural equation models. 

\section{Standard errors of standardized parameters}

Let $y$ be a $p$-vector of observed variables, from which a sample is obtained.
The following SEM for $y$ is specified:
\begin{align}
\label{eq:lisrel_observed}
y &= \Lambda \n + \epsilon\\
\n &= B_0 \n + \zeta,\label{eq:lisrel_latent}
\end{align}
where $\n$ is a vector of unobserved variables, $\zeta$ is a vector of
disturbance terms and $\epsilon$ is a vector of measurement errors. Model
\ref{eq:lisrel_latent} implies the following model
$\Sigma_\n(\theta)$ for the variance-covariance matrix of the unobserved
variables as a function of a parameter vector $\theta$: 
\begin{equation}\label{eq:sigma_n}
    \Sigma_\n(\theta) = B^{-1} \Phi B^{-T},
\end{equation}
where $B \definedas I - B_0$ is positive definite, 
and $\Phi$ is the variance-covariance
matrix of $\zeta$. Model \ref{eq:lisrel_observed} can then be seen to imply the
following model $\Sigma_y(\theta)$ for the variance-covariance matrix of the
observed variables:
\begin{equation}\label{eq:sigma_y}
    \Sigma_y(\theta) = \Lambda B^{-1} \Phi B^{-T} \Lambda' + \Psi,
\end{equation}
where  $\Psi$ is the variance-covariance matrix of $\epsilon$. 
We assume throughout that both $\Sigma_y$ and $\Sigma_\n$ are positive
definite. In what follows we will write $\Sigma_.$ for $\Sigma_.(\theta)$ in
the interest of clarity.

The parameters of the model are collected in a parameter
vector $$\theta \definedas [\vec{\Lambda}, \vech{\Phi}, \vec B_0, \vech
\Psi].$$
Interest focuses not only on the parameter vector $\theta$, 
but also on the so-called ``standardized'' parameter vector, denoted
$\thetastan \definedas [\Lambdastan, \Bstan_0]$, where:
\begin{align}\label{eq:lambda_s}
    \Lambdastan &\definedas D^{-1}_y \Lambda D_\n
    \\
    \Bstan_0 &\definedas D^{-1}_\n B_0 D_\n,\label{eq:beta_s}
\end{align}
and $D_y \definedas \sqrt{I \hadaprod \Sigma_y}$, and 
$D_\n \definedas \sqrt{I \hadaprod \Sigma_\n}$. 

%\marginpar{TODO: Define $\thetastan$.}

By standard application of the Delta method \citep[e.g.][]{oehlert1992note}, the asymptotic variance of $\thetastan$ can be shown to equal
\begin{equation}\label{eq:deltamethod}
	\left(\frac{d \thetastan}{d \theta}\right) 
		\var(\theta) 
	\left(\frac{d \thetastan}{d \theta}\right)'.
\end{equation}
Here $\var(\theta)$ is the appropriate asymptotic variance matrix of the free model parameters $\theta$ \cite[e.g.][]{satorra1989alternative}.
The choice of the appropriate $\var(\theta)$ matrix allows for incorporation of 
non-normal and complex sample data with possibly missing observations.



\subsection{Derivatives of the standardized parameters}


The derivatives $d \thetastan / d \theta$ in equation \ref{eq:deltamethod} are not available in the literature and are derived here.

From definition \ref{eq:lambda_s}, 
\begin{multline}\label{eq:dveclam}
\d\vec\Lambdastan = 
    (D_\n \Lambda' \kronprod I_p) \d \vec D_y^{-1} + 
    (D_\n \kronprod D_y^{-1}) \d\vec\Lambda + \\
    (I_q \kronprod D_y^{-1} \Lambda) \d\vec D_\n,
\end{multline}
and from definition \ref{eq:beta_s}, 
\begin{multline}\label{eq:dvecbeta}
\d\vec \Bstan_0 = 
    (D_\n B_0' \kronprod I_p) \d \vec D_\n^{-1} + 
    (D_\n \kronprod D_\n^{-1}) \d\vec B_0 + \\
    (I_q \kronprod D_\n^{-1} B_0) \d\vec D_\n.
\end{multline}

The differentials of the standardized parameter matrices are, thus, 
functions of the differentials of the covariance structure models $\Sigma_y$
and
$\Sigma_\n$.
\cite{neudecker1991linear} derived the differential of the implied variance
matrix $\Sigma_y$ of the observed variables. To ensure completeness of the treatment, we repeat it here:
\begin{equation}\label{eq:delta_y}
\begin{split}
\d\vec \Sigma_y = (I + K) (\Lambda B^{-1} \Phi B^{-T} \kronprod I) 
\d\vec\Lambda 
+ \\
(I + K) (\Lambda B^{-1} \Phi B^{-T} \kronprod \Lambda B^{-1}) \d\vec B_0
+ \\
(\Lambda B^{-1} \kronprod \Lambda B^{-1}) P \d \vech \Phi
+ \\
\d \vech \Psi,
\end{split}
\end{equation}
where the commutation matrix $K$, the duplication matrix $P$, and the operators
$\vec$ and $\vech$ are defined in \cite{magnus1988matrix}
Let $\Delta_\l$ equal the coefficient matrix
 of $\d\vec\Lambda$ in equation \ref{eq:delta_y}, 
$\Delta_\b$ the coefficient  of $\d\vec B_0$, and
$\Delta_\p$ the coefficient  of $\d\vech\Phi$.


The differential of the variance matrix of $\n$ can be obtained as:
\begin{equation}\label{eq:delta_n}
\begin{split}
\d\vec \Sigma_\n = 
(I + K) (B^{-1} \Phi B^{-T} \kronprod  B^{-1}) \d\vec B_0
+ \\
(B^{-1} \kronprod  B^{-1}) P \d \vech \Phi.
\end{split}
\end{equation}
Let $\Delta^*_\b$ and $\Delta^*_\p$  be the coefficients of $\d\vec B_0$ and
$\d\vech\Phi$ respectively in equation \ref{eq:delta_n}.

Also, let
\begin{align}
    C_{y2} \definedas&
        - (I_p \kronprod \frac{1}{2 (I_p \hadaprod \Sigma_y)^{\frac{3}{2}}})
        \diag[\vec(I_p)],
\\
    C_{\n1} \definedas&
        (I_q \kronprod \frac{1}{2 (I_q \hadaprod \Sigma_\n)^{\frac{1}{2}}})
        \diag[\vec(I_q)],
\\
    C_{\n2} \definedas&
        - (I_q \kronprod \frac{1}{2 (I_q \hadaprod \Sigma_\n)^{\frac{3}{2}}})
        \diag[\vec(I_q)].
\end{align}

Then, applying standard operations on \ref{eq:dveclam} and rearranging terms,
the differential of the standardized $\Lambdastan$ matrix is
\begin{equation}\label{eq:dveclam_final}
\begin{split}
  \d\vec\Lambdastan = 
     & [(D_\n \Lambda' \kronprod I_p) C_{y2} \Delta_\l + 
        (D_\n \kronprod D_y^{-1})] 
        \d\vec\Lambda +\\
     & [(D_\n \Lambda' \kronprod I_p) C_{y2} \Delta_\b + 
        (I_q \kronprod D_y^{-1}\Lambda) C_{\n1} \Delta^*_\b ] 
        \d\vec B_0 +\\
     & [(D_\n \Lambda' \kronprod I_p) C_{y2} \Delta_\p + 
        (I_q \kronprod D_y^{-1}\Lambda) C_{\n1} \Delta^*_\p ] 
        \d\vec \Phi +\\
     & (D_\n \Lambda' \kronprod I_p) C_{y2} 
        \d\vec \Psi.
\end{split}\end{equation}
And similarly, the differential of the standardized $\Bstan_0$ matrix from
equation \ref{eq:dvecbeta} is
\begin{equation}\label{eq:dvecbeta_final}
\begin{split}
  \d\vec \Bstan_0 =
     & [\{(D_\n B_0' \kronprod I_q) C_{\n2}  + 
        (I_q \kronprod D_\n^{-1}B_0) C_{\n1} \} \Delta^*_\b +
           (D_\n \kronprod D_\n^{-1})  ] 
        \d\vec B_0 +\\
     & [(D_\n B_0' \kronprod I_q) C_{\n2}  + 
        (I_q \kronprod D_\n^{-1}B_0) C_{\n1} ] \Delta^*_\p 
        \d\vec \Phi.
\end{split}\end{equation}

From the differentials in equations \ref{eq:dveclam_final} and
\ref{eq:dvecbeta_final},
we conclude that the derivative matrices  $G_{\tilde{\l}}$ and 
$G_{\tilde{\b}}$ of the standardized parameters $\vec \Bstan$ and 
$\vec \Lambdastan$ with respect to the free parameters of the model 
$\theta$ will be the partitioned matrices
\begin{multline}
\frac{d \tilde\l}{d \theta} = [
        (D_\n \Lambda' \kronprod I_p) C_{y2} \Delta_\l + 
            (D_\n \kronprod D_y^{-1}), \\
        (D_\n \Lambda' \kronprod I_p) C_{y2} \Delta_\p +
         (I_q \kronprod D_y^{-1}\Lambda) C_{\n1} \Delta^*_\p, \\
        (D_\n \Lambda' \kronprod I_p) C_{y2} \Delta_\b + 
            (I_q \kronprod D_y^{-1}\Lambda) C_{\n1} \Delta^*_\b,\\
        (D_\n \Lambda' \kronprod I_p) C_{y2}  
 ] 
\end{multline}
and
\begin{multline}
\frac{d \tilde\b_0}{d \theta} = [
        \0,
        \{(D_\n B_0' \kronprod I_q) C_{\n2}  + 
            (I_q \kronprod D_\n^{-1}B_0) C_{\n1} \} \Delta^*_\p
            ,\\
        \{(D_\n B_0' \kronprod I_q) C_{\n2}  + 
            (I_q \kronprod D_\n^{-1}B_0) C_{\n1} \} \Delta^*_\b +
            (D_\n \kronprod D_\n^{-1})
            ,
        \0
 ] .
\end{multline}

\marginpar{TODO: Add $\Phi$ and $\Psi$; put it all together in one matrix}

\section{Application: standard errors of standardized parameters}

Estimating the 

\begin{table}[ht]
\begin{center}
\begin{tabular}{lllcccc}  \hline  \hline
&&&  \multicolumn{4}{c}{\emph{Year}}\\\cline{4-7}
&&$n$& 2007&2008&2009&2010\\
%&	&	Primary		&VMBO		&		MBO	&HAVO/VWO&	HBO	&	WO	\\	
  \hline
  \multicolumn{2}{l}{Education level}\\
& Primary	   & $279$  & 0.799 (0.029) & 0.818 (0.024) & 0.828 (0.023) & 0.816 (0.027) \\ 
& Lower secondary  & $940$  & 0.821 (0.014) & 0.821 (0.013) & 0.812 (0.014) & 0.822 (0.014) \\ 
& Middle secondary & $782$  & 0.891 (0.016) & 0.879 (0.017) & 0.896 (0.015) & 0.877 (0.018) \\ 
& Upper secondary  & $369$  & 0.822 (0.015) & 0.820 (0.014) & 0.808 (0.015) & 0.805 (0.017) \\ 
& Lower tertiary   & $799$  & 0.869 (0.013) & 0.878 (0.012) & 0.863 (0.013) & 0.874 (0.013) \\ 
& Upper tertiary   & $256$  & 0.896 (0.017) & 0.898 (0.017) & 0.886 (0.018) & 0.887 (0.019) \\ 
  \hline     \hline
\end{tabular}
\caption{Reliability of self-rated health in the Netherlands 2007-2010 for different educational groups. 
Asymptotic standard errors based on the method described in this paper are also given. For direct tests of 
differences between the groups in reliability, see table \ref{tab:tests}.}\label{tab:health-education}
\end{center}
\end{table}


% latex table generated in R 2.13.0 by xtable 1.5-6 package
% Sun May 15 22:42:47 2011
\begin{table}[ht]
\begin{center}
\begin{tabular}{lrrrl}
  \hline  \hline
 Comparison & $t_{\mathrm{dif}}$ & $p_{\mathrm{dif}}$ & $p_{\mathrm{dif,adj}}$ &  \\ 
  \hline
Primary $<$$>$ vmbo & -0.68 & 0.25 & 0.96 &  \\ 
Primary $<$$>$ havo/vwo & -2.76 & 0.00 & 0.03 & * \\ 
Primary $<$$>$ mbo & -0.71 & 0.24 & 0.96 &  \\ 
Primary $<$$>$ hbo & -2.18 & 0.01 & 0.10 &  \\ 
Primary $<$$>$ wo & -2.86 & 0.00 & 0.02 & * \\ 
vmbo $<$$>$ havo/vwo & -3.25 & 0.00 & 0.01 & * \\ 
vmbo $<$$>$ mbo & -0.06 & 0.48 & 0.96 &  \\ 
vmbo $<$$>$ hbo & -2.46 & 0.01 & 0.06 &  \\ 
vmbo $<$$>$ wo & -3.35 & 0.00 & 0.01 & * \\ 
 havo/vwo $<$$>$ mbo & 3.12 & 0.00 & 0.01 & * \\ 
 havo/vwo $<$$>$ hbo & 1.07 & 0.14 & 0.71 &  \\ 
 havo/vwo $<$$>$ wo & -0.22 & 0.41 & 0.96 &  \\ 
 mbo $<$$>$ hbo & -2.33 & 0.01 & 0.08 &  \\ 
 mbo $<$$>$ wo & -3.22 & 0.00 & 0.01 & * \\ 
 hbo $<$$>$ wo & -1.26 & 0.10 & 0.62 &  \\ 
   \hline  \hline
\end{tabular}\caption{Multiple comparison between educational groups for the reliability in 2007. 
	The groups that are compared on reliability are shown in the first column. The second column shows the
	$t$-value for the test of no difference between the groups in reliability. The last two columns show 
	the corresponding $p$-value and $p$-value adjusted for multiple comparisons \citep{holm1979simple}, respectively.}\label{tab:tests}
\end{center}
\end{table}


\section{Discussion}

\subsection{Scope}

Limited, categorical variables, effects/correlations between latent variables,
factor loadings, error correlations.

\subsection{Others}
\cite{bollen1990direct} note that an approximate method exists for regression
analysis that does not take the estimation of the model-implied standard
deviations into account. 

References to Delta method but without providing the derivatives.

Likelihood based confidence intervals, bootstrapping.

Our method dispenses with the need for approximate methods or numerical
derivatives.

\subsection{Future}

Examine relative performance of confidence intervals, similar to work done for
indirect effects.


\bibliography{derivatives}


\appendix

\section{Calculations for the example analysis}


\begin{table}
\begin{tabular}{lrrrrrrrrrrrr}\hline\hline
	& \multicolumn{11}{c}{Unstandardized free parameters $\theta$}\\
	%\cline{2-12}
			& $\psi_{11}$ & $\psi_{22}$ &$\psi_{33}$ &$\psi_{44}$
			& $\phi_{11}$ & $\phi_{22}$ &$\phi_{33}$ &$\phi_{44}$ 
			& $\beta_{21}$ & $\beta_{32}$ &$\beta_{43}$
			\\
\hline
%\multicolumn{4}{l}{$d \vec \Lambda_s / d \vec \theta$} &\\
                   \\                                                      
$\lambda_{s, 11}$ &-0.733  & .      & .      & .     & 0.277 & .     & .     & .     & .     & .     & .   & \\
$\lambda_{s, 22}$&.     & -0.739  & .      & .     & 0.217 & 0.282 & .     & .     & 0.209 & .     & .   & \\
$\lambda_{s, 33}$&.      & .     & -0.755  & .     & 0.188 & 0.244 & 0.300 & .     & 0.180 & 0.225 & .   & \\
$\lambda_{s, 33}$&.      & .      & .     & -0.754 & 0.173 & 0.225 & 0.276 & 0.298 & 0.166 & 0.207 & 0.230\\
\\
%\multicolumn{4}{l}{$d \vec B_s / d \vec \theta$} &\\
             \\                                            
$\beta_{s, 21}$ &. & . & . & . & 0.232 & -1.059  & .      & .     & 0.223 & .     & .    & \\
$\beta_{s, 32}$&. & . & . & . & 0.131  & 0.171 & -1.147  & .     & 0.126 & 0.157 & .    & \\
$\beta_{s, 43}$&. & . & . & . & 0.058  & 0.076  & 0.093 & -1.193 & 0.056 & 0.070 & 0.077\\
\hline\hline
\end{tabular}
\caption{Derivatives of the standardized parameters $\Lambda_s$ and $B_s$ with respect
to the free parameters of the model in the quasi simplex example.}
\label{tab:jacobian}
\end{table}


\begin{table}
\begin{small}
\begin{tabular}{rrrrrrrrr}\hline\hline
	&$\psi$ & $\phi_{11}$ & $\phi_{22}$ &$\phi_{33}$ &$\phi_{44}$ & $\beta_{21}$ & $\beta_{32}$ &$\beta_{43}$  \\
\hline
  $\psi$   & \num{0.2e-4}\\
  $\phi_{11}$  & -\num{0.2e-4}  & \num{2.2e-4}\\
  $\phi_{22}$  & -\num{0.3e-4}  & \num{0.4e-4}  & \num{0.9e-4}\\
  $\phi_{33}$  & -\num{0.2e-4}  & \num{0.2e-4}  & \num{0.1e-4}  & \num{0.5e-4}\\
  $\phi_{44}$  & -\num{0.3e-4}  & \num{0.3e-4}  & \num{0.3e-4}  & \num{0.1e-4}  & \num{0.8e-4}\\
  $\beta_{21}$  & \num{0.4e-4} & -\num{1.5e-4} & -\num{0.9e-4} & -\num{0.1e-4} & -\num{0.5e-4}  & \num{4.3e-4}\\
  $\beta_{32}$  & \num{0.2e-4} & -\num{0.2e-4} & -\num{0.4e-4} & -\num{0.4e-4} & -\num{0.1e-4} & -\num{0.9e-4}   & \num{3.3e-4}\\
  $\beta_{43}$  & \num{0.3e-4} & -\num{0.3e-4} & -\num{0.3e-4} & -\num{0.3e-4} & -\num{0.7e-4}  & \num{0.4e-4}  & -\num{1.0e-4}   & \num{3.4e-4}\\
\hline\hline
\end{tabular}\end{small}
\caption{Variance-covariance matrix of the parameter estimates.}
\label{tab:vcov}
\end{table}

\vspace{12pt}

\begin{table}\begin{center}
\begin{tabular}{rrrr}\hline\hline
 Parameter & 	Estimate &	s.e. &	$z$\\
 \hline
$\lambda_{s,11}$ & 0.852& 0.006 &  134\\
$\lambda_{s,22}$ &0.851 &0.006  & 139\\
$\lambda_{s,33}$ &0.846 &0.006  & 136\\
$\lambda_{s,44}$ &0.847& 0.007  & 128\\
$\beta_{s,21}$& 0.882           &0.012 & 75\\
$\beta_{s,32}$& 0.920            &0.009 &103\\
$\beta_{s,43}$& 0.960            &0.011  &85\\
\hline\hline
\end{tabular}
\caption{Standardized parameter estimates and asymptotic standard errors for the example ($n = 3425$).}
\label{tab:vcov}\end{center}
\end{table}
\if 1=2
\begin{table}[ht]
\begin{center}
\begin{tabular}{rrrrrrrrrrrrrrr}
  \hline
 & hlt.2007 & hlt.2008 & hlt.2009 & hlt.2010 &  & hlt.2007 & hlt.2008 & hlt.2009 & hlt.2010 &  & hlt.2007 & hlt.2008 & hlt.2009 & hlt.2010 \\ 
  \hline
1 & 1.00 &  &  &  &  & 1.00 &  &  &  &  & 1.00 &  &  &  \\ 
  2 & 0.59 & 1.00 &  &  &  & 0.56 & 1.00 &  &  &  & 0.65 & 1.00 &  &  \\ 
  3 & 0.61 & 0.67 & 1.00 &  &  & 0.52 & 0.61 & 1.00 &  &  & 0.62 & 0.71 & 1.00 &  \\ 
  4 & 0.59 & 0.59 & 0.67 & 1.00 &  & 0.53 & 0.63 & 0.69 & 1.00 &  & 0.54 & 0.64 & 0.73 & 1.00 \\ 
  5 & 2.83 & 2.91 & 2.84 & 2.82 &  & 2.98 & 3.08 & 3.01 & 2.97 &  & 3.21 & 3.24 & 3.27 & 3.20 \\ 
  6 & 0.69 & 0.73 & 0.74 & 0.72 &  & 0.73 & 0.73 & 0.71 & 0.73 &  & 0.80 & 0.76 & 0.81 & 0.75 \\ 
  7 &  &  &  &  &  &  &  &  &  &  &  &  &  &  \\ 
  8 & 1.00 &  &  &  &  & 1.00 &  &  &  &  & 1.00 &  &  &  \\ 
  9 & 0.65 & 1.00 &  &  &  & 0.65 & 1.00 &  &  &  & 0.69 & 1.00 &  &  \\ 
  10 & 0.59 & 0.62 & 1.00 &  &  & 0.57 & 0.67 & 1.00 &  &  & 0.60 & 0.74 & 1.00 &  \\ 
  11 & 0.58 & 0.60 & 0.63 & 1.00 &  & 0.51 & 0.60 & 0.68 & 1.00 &  & 0.59 & 0.70 & 0.74 & 1.00 \\ 
  12 & 3.17 & 3.18 & 3.16 & 3.10 &  & 3.28 & 3.28 & 3.22 & 3.22 &  & 3.33 & 3.35 & 3.32 & 3.29 \\ 
  13 & 0.75 & 0.74 & 0.72 & 0.72 &  & 0.73 & 0.76 & 0.72 & 0.74 &  & 0.86 & 0.86 & 0.82 & 0.83 \\ 
   \hline
\end{tabular}
\end{center}
\caption{Correlations between the observed variables in each of the six groups of educational level.}
\end{table}
\fi



\section{Extension to multigroup SEM}

In multigroup SEM, let each matrix in the above equations equal the block
diagonal of the corresponding matrices in each group. 

Then define the stacking matrix $Z_{p \times pg} \definedas \left[I_p, I_p,
..., I_p\right]$. The derivatives are then 
$$
	(I_q \kronprod Z) G
$$

\end{document}
